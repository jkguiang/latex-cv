\documentclass[letterpaper,11pt]{article}

\usepackage{latexsym}
\usepackage[empty]{fullpage}
\usepackage{titlesec}
\usepackage{marvosym}
\usepackage[usenames,dvipsnames]{color}
\usepackage{verbatim}
\usepackage{enumitem}
\usepackage{bibentry}
\makeatletter\let\saved@bibitem\@bibitem\makeatother
\usepackage[pdftex]{hyperref}
\makeatletter\let\@bibitem\saved@bibitem\makeatother
\usepackage{multibib}
\usepackage{fancyhdr}
\usepackage{fontawesome}

\PassOptionsToPackage{fontawesome}{hyperref}
\pagestyle{fancy}
\fancyhf{} % clear all header and footer fields
\fancyfoot{}
\renewcommand{\headrulewidth}{0pt}
\renewcommand{\footrulewidth}{0pt}

% Adjust margins
\addtolength{\oddsidemargin}{-0.5in}
\addtolength{\evensidemargin}{-0.5in}
\addtolength{\textwidth}{1in}
\addtolength{\topmargin}{-.5in}
\addtolength{\textheight}{1.0in}

\urlstyle{same}

\raggedbottom
\raggedright
\setlength{\tabcolsep}{0in}

% Sections formatting
\titleformat{\section}{
  \vspace{-4pt}\scshape\raggedright\large
}{}{0em}{}[\color{black}\titlerule \vspace{-5pt}]

%-------------------------
% Custom commands
\newcommand{\Item}[3]{
  \item\small{
    \ifx&#1&
      \textbf{#2}{: #3 \vspace{-2pt}}
    \else
      \textbf{\href{#1}{#2}}{: #3 \vspace{-2pt}}
    \fi
  }
}

\newcommand{\Subheading}[4]{
  \vspace{-1pt}\item
    \begin{tabular*}{0.97\textwidth}{l@{\extracolsep{\fill}}r}
      \textbf{#1} & #2 \\
      \textit{\small#3} & \textit{\small #4} \\
    \end{tabular*}\vspace{-5pt}
}

\newcommand{\SubItem}[3]{\Item{#1}{#2}{#3}\vspace{-4pt}}

\newcommand{\Section}[2]{
    \section{\texorpdfstring{#1}{o} #2}
}

\renewcommand{\labelitemii}{$\circ$}

\newcommand{\SubHeadingListStart}{\begin{description}[leftmargin=*]}
\newcommand{\SubHeadingListEnd}{\end{description}}
\newcommand{\ItemListStart}{\begin{itemize}}
\newcommand{\ItemListEnd}{\end{itemize}\vspace{-5pt}}

%-------------------------------------------
%%%%%%  CV STARTS HERE  %%%%%%%%%%%%%%%%%%%%%%%%%%%%


\begin{document}

%----------HEADING-----------------
\begin{tabular*}{\textwidth}{l@{\extracolsep{\fill}}r}
  \textbf{\href{https://www.jguiang.com}{\Large Jonathan Guiang}} & Email: \href{mailto:jguiang@ucsd.edu}{jguiang@ucsd.edu}\\
  \textit{\small{Physicist, Programmer, Data Analyst}} & Phone : +1-858-880-5819 \\
\end{tabular*}
\begin{center}
    \small{\href{https://www.jguiang.com}{\faGlobe\ jguiang.com} $|$ \href{https://www.github.com/jkguiang}{\faGithub\ github.com/jkguiang} $|$ \href{https://www.linkedin.com/in/jonathanguiang}{\faLinkedinSquare\ /in/jonathanguiang/}}
\end{center}
%

%--------SKILLS------------
\Section{\faSliders}{Skills}
 \SubHeadingListStart
    \SubItem{}{Languages}{Python, C++, Javascript, Matlab}
    \SubItem{}{Technologies}{ROOT, Pandas, Tensorflow, XGBoost, React, JQuery, Flask, Docker, Spark SQL, OpenSCAD, HTML, CSS, PHP, Mathematica, Git}

 \SubHeadingListEnd
%

%-----------EDUCATION-----------------
\Section{\faGraduationCap}{Education}
  \SubHeadingListStart
    \Subheading
      {UC San Diego}{San Diego, CA}
      {PhD. in Physics;  GPA: 3.79}{Sep. 2019 -- Present}
      \ItemListStart
        \Item{}{Dissertation}
            {TBA}
        \Item{}{Relevant Coursework}
          {Mathematical Methods in Physics, Quantum Field Theory I, Computational Physics}
        \Item{}{Awards}
          {Physics Excellence Award}
      \ItemListEnd
      
    \Subheading
      {UC Santa Barbara}{Santa Barbara, CA}
      {B.S. in Physics;  GPA: 3.73}{Sep. 2015 -- June. 2019}
      \ItemListStart
        \Item{https://www.physics.ucsb.edu/sites/default/files/sitefiles/education/thesis/Guiang.pdf}{Senior Thesis}
          {MTD Simulation and Search for Rare Higgs Decays}
        \Item{https://www.physics.ucsb.edu/education/undergrad/department-honors}{Awards}
          {Highest Academic Honors, Research Excellence Award, Distinction in the Major}
      \ItemListEnd
  \SubHeadingListEnd
%

%-----------FELLOWSHIPS-----------------
\Section{\faUniversity}{Fellowships}
  \SubHeadingListStart
    \SubItem{http://sloanphds.org/mphd.html}{Sloan MPHD Scholar}
      {Named an Alfred P. Sloan Foundation’s Minority Ph.D. (MPHD) Scholar in 2019-20. Served as an official mentor to incoming Scholars.}
  \SubHeadingListEnd
%

%-----------EXPERIENCE-----------------
\Section{\faSuitcase}{Experience}
  \SubHeadingListStart
 
     \Subheading
      {UC San Diego}{San Diego, CA}
      {Graduate Student Researcher, CMS Tier 2 Data Manager}{Apr. 2019 -- Present}
      \SubItem{}{Experimental Particle Physics}{Performed data analysis on proton-proton collision data collected with the CMS detector at the LHC as a part of a multinational collaboration. Listed as an author under ``CMS Collaboration" on all CMS publications since Fall, 2020. Managed 3 petabytes of CMS data stored at the UCSD Tier 2 computing facility.}
      \ItemListStart
        \Item{}{VBS WWH Analysis}
          {Measurement of the yet unprobed trilinear Higgs boson self coupling through the Vector Boson Scattering (VBS) production mechanism. Developed background estimation techniques and contributed to the software framework for this analysis and related analyses.}
      \ItemListEnd
      \SubItem{}{Radiology}{Worked with an interdisciplinary group of researchers using state-of-the-art technology to conceptualize and refine machine learning algorithms for analyzing and clinically utilizing CT scans.}
      \ItemListStart
        \Item{}{Covid-19 Pneumonia}
          {Trained a DenseUNet to segment pneumonia in CT scans. Analyzed resultant segmentation masks for diagnostic and prognostic interests.}
        \Item{}{Lung Nodules}
          {Trained a 3D Convolutional Neural Network to classify annotated lung nodules on CT scans as malignant or benign for use as a biopsy screening tool in the clinic.}
      \ItemListEnd
  
    \Subheading
      {CERN-HEP Software Foundation}{San Diego, CA}
      {Student Developer}{May 2019 -- Aug. 2019}
      \ItemListStart
        \Item{https://summerofcode.withgoogle.com/projects/5810325671116800}{CMS Data Access}
          {Developed open source software for CERN-HSF with funding from Google Summer of Code. Contributed to the maintenance effort of a caching infrastructure used by scientists across the continental U.S. Produced a set of tools for cleansing, extracting, and visualizing cache access pattern data. Analyzed and presented insights provided by these tools in order to demonstrate their effectiveness}
      \ItemListEnd

    \Subheading
      {UC Santa Barbara}{Santa Barbara, CA}
      {Undergraduate Student Researcher}{Dec. 2016 -- Jun. 2019}
      \ItemListStart
        \Item{https://github.com/jkguiang/rare-higgs}{Rare Higgs Decay Analysis}
          {Measurement of $H \rightarrow \rho/\phi+\gamma$ decays using events from a sample of proton-proton collisions collected with the CMS detector, where anomalous decay rates would indicate existence of new physics. Designed and implemented this novel analysis from the ground up using a numpy/pandas framework.}
        \Item{https://github.com/jkguiang/Chronosim}{MIP Timing Detector (MTD)}
          {Developed software for optimizing the design for the MTD to be constructed for the HL-LHC. Used simulated particle kinematics in addition to a tunable OpenSCAD 3D model of the sensor layout to simulate the efficiency of the detector in direct collaboration with the team responsible for its construction.}
        \Item{https://github.com/jkguiang/AutoDQM}{AutoDQM}
          {Conceptualized, designed, and implemented a statistical tool for data quality management with an online graphical interface for ease of use. Collaborated with a computer science student in Switzerland to further improve the platform and market it to other research groups.}
        \Item{https://github.com/bjmarsh/pmt-calibration}{MilliQan}
          {Characterized the single-photoelectron (SPE) response of photomultiplier tubes used in the “MilliQan” experiment demonstrator under the direction of a graduate student, working closely with another undergraduate. Developed software for simulating SPE responses.}
      \ItemListEnd

  \SubHeadingListEnd
%

%-----------PROJECTS-----------------
\Section{\faCode}{Projects}
  \SubHeadingListStart
    \SubItem{https://github.com/jkguiang/rapido}{RAPIDO}
      {Repeatable Analysis Programming for Interpretability, Durability, and Organization (RAPIDO). C++ framework for performing High Energy Particle Physics data analysis.}
    \SubItem{https://www.integratable.info}{Integratable}
      {A public tool that provides useful integrals on an interactive, modern platform. Uses a React-based frontend, evaluates known definite integrals using Javascript mathematics functions.}
    \SubItem{https://www.jguiang.com}{Personal Website}
      {Simple website built using React and deployed on github pages. Used assets from Font Awesome and Bootstrap and added animations using react-pose.}
    \SubItem{https://devpost.com/software/chompchap}{ChompChapp}
      {Made for the SB Hacks V Hackathon and selected as one of the top six projects of the event. Made intelligent restaurant suggestions based on subconscious user preference. Javascript/JQuery webpage served by Flask/Celery on a Redis server. Powered by a Python backend with Tensorflow and Keras machine learning models.}
  \SubHeadingListEnd
%

%-----------COMMUNITY-----------------
\Section{\faGroup}{Community}
  \SubHeadingListStart
      \Subheading
      {EXPAND}{San Diego, CA}
      {Co-founder}{January 2020 -- Present}
      \ItemListStart
        \Item{https://center.ucsd.edu/programs/mentorship-program.html}{Description}
          {EXperiential Projects for Accelerated Networking and Development (EXPAND).  Program pairs an undergraduate mentee with a graduate-researcher mentor. The mentor/mentee pair work towards the completion of a project sourced from or contributing to the mentor's research and honed to target industry-specific skills. Produced in collaboration with the UCSD Physical Sciences Student Success Center.}
        \Item{}{Positions}
            {Program Organizer (Duration), Graduate Student Mentor (Spring, 2021)}
      \ItemListEnd
      \Subheading
      {Physics Graduate Council}{San Diego, CA}
      {Representative}{September 2020 -- Present}
      \ItemListStart
        \Item{https://sites.google.com/a/physics.ucsd.edu/physics-graduate-council/}{Description}
          {The Physics Graduate Council (PGC) is a body of elected representatives which serves the interests of the physics graduate students of UCSD. It serves as the official liaison between the department administration and faculty and the graduate student body. Nominated as an  additional volunteer representative for the 2020 to 2021 academic year in order to better organize an ongoing graduate student diversity initiative.}
        \Item{https://sites.google.com/a/physics.ucsd.edu/physics-graduate-council/representatives}{Positions}
            {Representative (Duration), Department Climate Committee Member (Duration)}
      \ItemListEnd
  \SubHeadingListEnd
%

%-----------PUBLICATIONS-----------------
\nobibliography{publications}
\bibliographystyle{JHEP3}
\Section{\faPencilSquareO}{Publications}
  \begin{itemize}
      \item \bibentry{ball2020search}
  \end{itemize}

\Section{\faArchive}{Proceedings}
  \begin{itemize}
      \item \bibentry{fajardo2020moving}
  \end{itemize}
  
% \Section{\faMicrophone}{Presentations}
%   \begin{itemize}
%       \item test
%   \end{itemize}
%-------------------------------------------
\end{document}
